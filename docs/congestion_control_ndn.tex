\documentclass[10pt]{article}
\usepackage{amsmath,amssymb,epsfig,graphics,hyperref,amsthm,mathtools,
  mdframed,enumitem}
\DeclarePairedDelimiter\ceil{\lceil}{\rceil}
\DeclarePairedDelimiter\floor{\lfloor}{\rfloor}

\hypersetup{colorlinks=true}

\setlength{\textwidth}{7in}
\setlength{\topmargin}{-0.575in}
\setlength{\textheight}{9.25in}
\setlength{\oddsidemargin}{-.25in}
\setlength{\evensidemargin}{-.25in}

\reversemarginpar
\setlength{\marginparsep}{-15mm}

\newcommand{\rmv}[1]{}
\newcommand{\bemph}[1]{{\bfseries\itshape#1}}
\newcommand{\N}{\mathbb{N}}
\newcommand{\Z}{\mathbb{Z}}
\newcommand{\imply}{\to}
\newcommand{\bic}{\leftrightarrow}

% Some user defined strings for the homework assignment
%
\def\Author{Shuo Yang}

\begin{document}

\noindent

% \CourseCode \hfill \DateHandedOut

\begin{center}
  Consumer Driven Congestion Control Design for NDN\\
  \vspace{1em}
\end{center}

% A horizontal split line
\hrule\smallskip
\vspace{1em}
\underline{\textbf{Design}}\\

In this design, congestion can be detected in three different ways:
\begin{enumerate}
\item Re-transmission timer expires.\\
  \emph{action:} Reset the current window size to 1.
\item A NACK received with the reason as ``congestion''.\\
  If the sequence number of the received NACK is
  greater than the sequence number of the last received NACK $+$ $k *
  $(the current window size), where $k<1$. the consumer should regard
  this as a sign of 
  congestion. The purpose of doing this is to avoid the sharp drop
  of the congestion window size caused by the situation where a
  consecutive sequence of NACKs were received because of a burst of
  packet drops occurred in the network. In such case, we want
  only one multiplicative decrease like what TCP did.\\
  \emph{action:} multiplicative decrease, fast re-transmission and
  fast recovery.
\item A hole in the packets sequence detected.\\
  When the consumer sends interest packets out, it records the
  order of the sent interests. When the data packets come back, the
  consumer can check whether the data packets are in the same order
  as the interests. The assumption is that most time data packets
  should arrive in order. If the consumer expected the $N_{th}$
  data packet to arrive, but instead the $M$ more data after it have
  arrived, the consumer regards this as a sign of congestion.\\
  \emph{action:} multiplicative decrease, fast re-transmission and
  fast recovery.
\end{enumerate}

\vspace{1em}
\underline{\textbf{Important Data Structures and Variables}}\\
\begin{mdframed}
  \textbf{sentQueue}: this queue is used to hold the interest packets that are in
  the current window, i.e., they have not been acknowledged yet. Once
  the consumer received the data packets, it removes the
  corresponding entries in the queue.\\\\
  \textbf{retxQueue}: this queue is used to hold the interest packets that
  should be re-transmitted due to congestion.\\\\
  \textbf{retxedList}: this list is used to hold the interest packets
  that have been re-transmitted and not acknowledged yet.\\\\
  \textbf{outOfOrderDataList}: this list is used to hold the
  received data packets that are out of order. As long as the consumer
  received the data packet with the expected sequence number, this
  list should be set to empty.\\\\
  \textbf{cwnd}: congestion window size\\
  \textbf{ssthresh}: slow start threshold\\
  \textbf{nextSeqNum}: the sequence number of the next interest packet
  to be sent out in order. This number should be monotonically
  increasing.\\
  \textbf{lastReceivedNack}: the most-recently received NACK packet.\\
  \textbf{RTO}: value of the current calculated RTO
\end{mdframed}

\vspace{1em}
\underline{\textbf{Adjustable Parameters}}\\
\begin{mdframed}
  \textbf{initialCwnd}: initial value for cwnd\\
  \textbf{initialSsthresh}: initial value for ssthresh\\
  \textbf{MDcoef}: coefficient for multiplicative decrease. In TCP,
  its value is 0.5\\
  \textbf{AIstep}: incremental quantity for additive increase. In TCP,
  its value is 1\\
  \textbf{maxOutOfOrderData}: the maximum number of consecutively
  arrived out-of-order data packets the consumer can withstand until
  triggering a congestion signal\\
  \textbf{rtoBackoffMultiplier}: the multiplier for RTO back off. In
  TCP, its value is 2\\
\end{mdframed}

\vspace{1em}
\underline{\textbf{Congestion Control Algorithm}}\\
\begin{mdframed}
  \underline{\textbf{Init()}}\\
  \-\hspace{1em} cwnd = initialCwnd\\
  \-\hspace{1em} ssthresh = initialSsthresh\\
  \-\hspace{1em} nextSeqNum = 1\\
  \-\hspace{1em} lastReceivedNack = null\\
  \-\hspace{1em} sentQueue = []\\
  \-\hspace{1em} retxQueue = []\\
  \-\hspace{1em} retxedList = []\\
  \-\hspace{1em} outOfOrderDataList = []\\
  \-\hspace{1em} maxOutOfOrderData = 3\\
  \-\hspace{1em} ScheduleNextInterest()  /$*$ send out the first
  interest $*$/ \\\\
  \underline{\textbf{ScheduleNextInterest()}} \\
  \-\hspace{1em} \textbf{if} !retxQueue.empty()\\
  \-\hspace{2em} SendInterest(retxQueue.first()) /$*$ send out interests
  that need re-transmission first $*$/\\
  \-\hspace{2em} retxedList.add(retxQueue.first())\\
  \-\hspace{2em} retxQueue.removeFirst()\\
  \-\hspace{1em} \textbf{else}\\
  \-\hspace{2em} SendInterest(nextSeqNum)\\
  \-\hspace{2em} nextSeqNum = nextSeqNum + 1\\\\
  \underline{\textbf{SendInterest(seqNum)}} \\
  \-\hspace{1em} sentQueue.add(seqNum)\\
  \-\hspace{1em} timer[seqNum] = RTO\\
  \-\hspace{1em} transmitInterest(seqNum)\\\\
  \underline{\textbf{OnData(data)}} \\
  \-\hspace{1em} \textbf{if} !retxedList.contains(data.seqNum) /$*$ do
  not take re-transmitted packets as RTT samples $*$/\\
  \-\hspace{2em} UpdateRTO(EstimateRTT(data.seqNum))\\
  \-\hspace{1em} \textbf{else}\\
  \-\hspace{2em} retxedList.remove(data.seqNum)\\
  \-\hspace{1em} \textbf{if} sentQueue.first() == data.seqNum
  /$*$ expected data packet $*$/\\
  \-\hspace{2em} outOfOrderData.clear()\\
  \-\hspace{2em} \textbf{if} cwnd $<$ ssthresh\\
  \-\hspace{3em} cwnd = cwnd + AIstep /$*$ additive increase $*$/ \\
  \-\hspace{2em} \textbf{else} \\
  \-\hspace{3em} cwnd = cwnd + AIstep/cwnd /* congestion avoidance */
  \\
  \-\hspace{1em} \textbf{else} /$*$ data arrived out of order $*$/ \\
  \-\hspace{2em} outOfOrderData.add(data.seqNum)\\
  \-\hspace{2em} \textbf{if} outOfOrderData.size() $>$
  maxOutOfDrderData /$*$ congestion signal $*$/\\
  \-\hspace{3em} ssthresh =  MDcoef $*$ cwnd\\
  \-\hspace{3em} cwnd = ssthresh /$*$ fast recovery $*$/ \\
  \-\hspace{3em} retxQueue.add(sentQueue.first())\\
  \-\hspace{3em} ScheduleNextInterest() /$*$ fast re-transmission $*$/\\
  \-\hspace{2em} \textbf{else}\\
  \-\hspace{3em} /$*$ don't increase cwnd here $*$/ \\\\
  \-\hspace{1em} sentQueue.remove(data.seqNum) /$*$ remove the
  acknowledged interest from the queue $*$/\\\\
  \underline{\textbf{OnNack(nack)}} \\
  \-\hspace{1em} \textbf{if} nack.seqNum $>$ lastReceivedNack.seqNum +
  cwnd\\
  \-\hspace{2em} ssthresh =  MDcoef $*$ cwnd\\
  \-\hspace{2em} cwnd = ssthresh /$*$ fast recovery $*$/ \\
  \-\hspace{2em} lastReceivedNack = nack /$*$ update $*$/ \\\\
  \-\hspace{1em} retxQueue.add(nack.seqNum)\\
  \-\hspace{1em} ScheduleNextInterest() /$*$ fast re-transmission $*$/\\\\
  \underline{\textbf{OnTimeout(seqNum)}} \\
  \-\hspace{1em} ssthresh =  MDcoef $*$ cwnd\\
  \-\hspace{1em} cwnd = 1\\
  \-\hspace{1em} RTO = rtoBackoffMultiplier * RTO\\\\
  \-\hspace{1em} retxQueue.add(seqNum)\\
  \-\hspace{1em} ScheduleNextInterest()
\end{mdframed}

\vspace{1em}
\underline{\textbf{RTT Estimation and RTO Calculation}}\\

\underline{\emph{Design Principles}}

\begin{itemize}[noitemsep]
\item Compliant with RFC 6298
  (\url{https://tools.ietf.org/html/rfc6298}).
\item Using Karn's algorithm, i.e., don't take RTT sample from
  re-transmitted packets.
\item Adjust parameters to common TCP defaults.
\end{itemize}

\underline{\emph{Variables and Parameters}}
\begin{mdframed}
  \textbf{RTO} : re-transmission timeout\\
  \textbf{SRTT} : smoothed round-trip time\\
  \textbf{RTTVAR} : round-trip time variation\\
  \textbf{G} : clock granularity (unit: seconds)\\
  \textbf{alpha} : RTT estimation filter gain\\
  \textbf{beta} : variance filter gain\\
  \textbf{MINRTO} : lower-bound of RTO\\
  \textbf{MAXRTO} : upper-bound of RTO\\
  \textbf{K} : coefficient for calculating RTO\\
\end{mdframed}

\underline{\emph{Initial Variable and Parameter Settings}}
\begin{mdframed}
  MINRTO = 1; //initially, the sender should set RTO to 1s, as suggested
  in RFC 6298\\
  RTO = MINRTO;\\
  alpha = 1/8; // suggested in RFC 6298\\
  beta = 1/4; // suggested in RFC 6298\\
  MAXRTO = 60; // RFC 6289 suggests this value should be at least 60s\\
  G = 0.1; // 0.1 second, RFC 6298 suggested that finer clock
  granularities ($<=$ 100ms) perform better than coarser
  granularities.
\end{mdframed}

\underline{\emph{RTT estimation and RTO calculation ($R$ is the
    current RTT measurement)}}
\begin{mdframed}
  \textbf{RttEstimation(R)}\\
  \-\hspace{2em} \textbf{if} it is the very first measurement:\\
  \-\hspace{4em} SRTT = R;\\
  \-\hspace{4em} RTTVAR = R/2;\\
  \-\hspace{2em} \textbf{else:} // subsequent RTT measurement\\
  \-\hspace{4em} RTTVAR = (1 - beta) * RTTVAR + beta * $|$SRTT - R$|$;\\
  \-\hspace{4em} SRTT = (1 - alpha) * SRTT + alpha * R;
\end{mdframed}

\emph{Note} that RTT sample should not be taken from the
re-transmitted packets according to Karn's algorithm.\\

\vspace{1em}
After computing SRTT and RTTVAR, sender should update RTO:\\

\-\hspace{2em} RTO = SRTT + max (G, K * RTTVAR)\\

\emph{Note} that if the computed RTO is less than 1 second, it should be
rounded up to 1 second, as suggested in RFC 6298.  RFC 6298
recommends a large minimum value on the RTO to keep TCP conservative
and avoid spurious re-transmission.

\vspace{2em}
\underline{\textbf{Algorithm for managing the re-transmission
    timer}}\\

\emph{note: the algorithm is recommended in RFC 6298, terminologies
  are adjusted for NDN. Also section (5.7) is specific to TCP's
  three-way handshake, so is not included.}
\begin{mdframed}
  \begin{enumerate}
  \item Every time an interest packet is sent (including a
    re-transmission), if the timer is not running, start it running so
    that it will expire after RTO seconds (for the current value of
    RTO).
  \item When the outstanding interest packets have been acknowledged
    (by their corresponding data packets), turn off the
    re-transmission timer.
  \item When a data packet is received that acknowledges a new interest
    packet, restart the re-transmission timer so that it will expire
    after RTO seconds (for the current value of RTO).\\
  \end{enumerate}

  When the re-transmission timer expires, do the following:
  \begin{enumerate}
  \item Re-transmit the earliest interest packet that has not been
    acknowledged yet.
  \item The consumer must set RTO = RTO * 2 (``exponentially back off
    the timer''). The value of MAXRTO may be used to provide an upper
    bound to this doubling operation.
  \item Start the re-transmission timer so that it will expire
    after RTO seconds (for the value of RTO after the doubling
    operation).
  \end{enumerate}
\end{mdframed}

\vspace{2em}

\end{document}
