\documentclass[10pt]{article}
\usepackage{amsmath,amssymb,epsfig,graphics,hyperref,amsthm,mathtools,
  mdframed,enumitem}
\DeclarePairedDelimiter\ceil{\lceil}{\rceil}
\DeclarePairedDelimiter\floor{\lfloor}{\rfloor}

\hypersetup{colorlinks=true}

\setlength{\textwidth}{7in}
\setlength{\topmargin}{-0.575in}
\setlength{\textheight}{9.25in}
\setlength{\oddsidemargin}{-.25in}
\setlength{\evensidemargin}{-.25in}

\reversemarginpar
\setlength{\marginparsep}{-15mm}

\newcommand{\rmv}[1]{}
\newcommand{\bemph}[1]{{\bfseries\itshape#1}}
\newcommand{\N}{\mathbb{N}}
\newcommand{\Z}{\mathbb{Z}}
\newcommand{\imply}{\to}
\newcommand{\bic}{\leftrightarrow}

% Some user defined strings for the homework assignment
%
\def\Author{Shuo Yang}

\begin{document}

\noindent

% \CourseCode \hfill \DateHandedOut

\begin{center}
  NDN File Transfer Application with Congestion Control\\
  Shuo Yang\\
  \vspace{1em}
\end{center}

% A horizontal split line
\hrule\smallskip
\vspace{1em}
\section{Development Environment Setup}

\subsection{Code repository}
\texttt{git@dyadis.cs.arizona.edu:ndn\_cc\_2016}\\

Repository structure:

\texttt{README.md}: step by step instructions on how to setup the development
environment

\texttt{apps/file-transfer/}: file transfer application

\texttt{docs/}: documentation

\texttt{scripts/}: scripts for installing virtual machine, running minindn
experiments, etc.

\texttt{results/}: experiment results

\subsection{Development environment setup}

The development environment I use is a Ubuntu 15.04 VM with NDN
related software (ndn-cxx, nfd and minindn) installed.\\\\
Please follow the steps in the \texttt{README.md} to set up the
development environment and how to compile the code and run it.\\\\
One thing we need to be aware of about minindn is that you must put the
experiment scripts into minindn's source code structure under
\texttt{mini-ndn/ndn/experiments} in order to run the experiments. And
every time we make changes to the scripts, we must reinstall minindn
to make them effective.

\section{Implementation}

\subsection{Queue size calculation}

First, we need to figure out how big a data packet is.
For a ndn data packet with content payload with the size of 1024 bytes,
\texttt{tcpdump} shows that the following:

\begin{mdframed}
07:47:39.952381  In 26:44:97:59:e6:cf (oui Unknown) ethertype IPv4
(0x0800), length 1449: (tos 0x0, ttl 64, id 6388, offset 0, flags
[none], proto UDP (17), length 1433)\\
    1.0.0.2.6363 $>$ 1.0.0.1.6363: [bad udp cksum 0x0799 $->$ 0xbb08!]
    UDP, length 1405
\end{mdframed}

Based on this, we can infer the packet structure as follows:

- total data packet size is 1449 bytes

- UDP packet size: 1433 bytes

- UDP payload (ndn data packet) size: 1405

- ethernet header: 1449 - 1433 = 16 bytes

- IP header + UDP header: 1433 - 1405 = 28 bytes

- ndn header overhead (signature, etc): 1405 - 1024 = 381 bytes\\

Next, I measured the RTT for a single pair of interest and data packet
(with content payload: 1024bytes) traveling through the linear
topology. I repeated the same process 5 times and got the average.

Below is the RTTs I got:

71ms

75ms

72.8ms

71.9ms

75.2ms

---------------

avg: 73.2ms\\

So the queue size is calculated as:
\begin{align*}
  queue\_size &= (rtt * bandwidth) / data\_packet\_size\\
  &= (73.2ms * 5Mbps) / 1499bytes\\
  &= 31.6 \# \text{of data packets}\\
  &\approx 32
\end{align*}

\subsection{AIMD implementation}

\begin{enumerate}
\item When a data packet was received, if $cwnd < ssthresh$, increment
  $cwnd$ by 1 (additive increase, slow start), otherwise increment
  $cwnd$ by $1/cwnd$ (linear increase, congestion avoidance).
\item When an interest packet was timed out, set $ssthresh$ to half of
  $cwnd$ (multiplicative decrease) and reset $cwnd$ to 1.
\end{enumerate}

\subsection{Sequence hole detection implementation}

This is implemented as an optional congestion control scheme and can
be turned on or off by a command line option. By default, it is turned
off.\\\\
The assumption for this scheme is that most time data packets should
arrive in order.

\begin{enumerate}
\item When a data packet was received, if it arrived in order (in the
  same order as the corresponding interest packet), then we use AIMD
  scheme as mentioned above to adjust the size of congestion window.
\item If it arrived out of order, we keep a counter on how many
  out-of-order data packets have received.
\item If the counter is below a certain threshold (5 by default), we
  don't adjust the size of congestion window.
\item If the counter is above the threshold (5 by default), we set
  $ssthresh$ to half of $cwnd$ and set $cwnd$ to $ssthresh$ (fast recovery).
\end{enumerate}

\subsection{Timer and timeout implementation}

\begin{enumerate}
\item For each sent interest, we keep track of the time it was
  sent and RTO value calculated at the time it was sent.
\item Every 10 ms, check RTO timers for each sent interest, if the
  timer expires, adjust the congestion window size according to the
  AIMD scheme and retransmit the timed out packet.
\end{enumerate}

\section{Experiment}

\subsection{Experiment setup}

\underline{Topology}\\\\
We are currently using a linear topology for debugging purpose:\\
consumer-----router1-----router2-----producer\\
host-router links (10Mbps, delay=10ms),  router-router link (5Mbps,
delay=10ms, max\_queue\_size=32)\\\\

\underline{File transfer applications}\\\\
We ran four different file transfer applications to download a file of
size 10MB, and each was repeated 5 times.
\begin{enumerate}
\item File transfer application with a fixed congestion window size.
\item File transfer application with AIMD scheme.
\item File transfer application with AIMD+Hole\_detection scheme.
\item FTP based on TCP/IP.
\end{enumerate}

\subsection{Parameters}
\begin{mdframed}
  MaxRTO = 10s; // maximum RTO value\\
  alpha = 1/8; // filter gain\\
  beta = 1/4; // filter gain\\
  RTT variance multiplier = 4\\
  RTO backoff multiplier = 2\\
  initial cwnd = 1\\
  initial ssthresh = 200\\
  additive increase step = 1\\
  multiplicative decrease factor = 0.5\\
  out-of-order packets counter threshold = 5\\
  retransmission timer check interval = 10ms
\end{mdframed}

\subsection{Metrics}
\begin{enumerate}
\item \emph{Download time}: total time (in seconds) it takes to download the file.
\item \emph{Actual Throughput}: calculated as (total number of data packets received *
  data packet size) / (total download time).
\item \emph{Effective Throughput}: calculated as (number of data packets received *
  data packet size) / (total download time).
\item \emph{Loss rate}: calculated as (total number of retransmitted packets) /
  (total number of packets received).
\item \emph{congestion window size}: measured every 10ms
\end{enumerate}

\section{Results}

\underline{NDN file transfer application with fixed cwnd=50}\\\\
\begin{tabular}{ | l | l | l | l | l |}
  \hline
  & download time (s) & actual throughput (kbps) & effective
  throughput (kbps) & loss rate \\\hline
  & 24.1 & 4932.18 & 4932.18 & 0.2\% \\\hline
  & 23.9 & 4964.96 & 4964.96 & 0.2\% \\\hline
  & 23.8 & 4981.83 & 4981.83 & 0.2\% \\\hline
  & 23.8 & 4980.4 & 4980.4 & 0.2\% \\\hline
  & 23.8 & 4987.08 & 4987.08 & 0.2\% \\\hline
\end{tabular}

\vspace{1em}
\underline{NDN file transfer application with fixed cwnd=100}\\\\
\begin{tabular}{ | l | l | l | l | l |}
  \hline
  & download time (s) & actual throughput (kbps) & effective
  throughput (kbps) & loss rate \\\hline
  & 25.1 & 4720.92 & 4720.92 & 5.6\% \\\hline
  & 25.5 & 4654.68 & 4654.68 & 5.7\% \\\hline
  & 25.6 & 4634.07 & 4634.07 & 5.6\% \\\hline
  & 26.8 & 4426.16 & 4426.16 & 5.6\% \\\hline
  & 24.6 & 4818.9 & 4818.9 & 5.6\% \\\hline
\end{tabular}

\vspace{1em}
\underline{NDN file transfer application with fixed cwnd=200}\\\\
\begin{tabular}{ | l | l | l | l | l |}
  \hline
  & download time (s) & actual throughput (kbps) & effective
  throughput (kbps) & loss rate \\\hline
  & 25.6 & 4640.69 & 4640.69 & 17.2\% \\\hline
  & 26.2 & 4520.9 & 4520.9 & 17.5\% \\\hline
  & 26.2 & 4520.48 & 4520.48 & 17.5\% \\\hline
  & 26.3 & 4514.49 & 4514.49 & 17.6\% \\\hline
  & 24.7 & 4811.88 & 4811.88 & 17.1\% \\\hline
\end{tabular}

\vspace{1em}
\underline{NDN file transfer application with AIMD scheme}\\\\
\begin{tabular}{ | l | l | l | l | l |}
  \hline
  & download time (s) & actual throughput (kbps) & effective
  throughput (kbps) & loss rate \\\hline
  & 50 & 4290.2 & 2373.14 & 100\% \\\hline
  & 46.5 & 4520.15 & 2551.64 & 100\% \\\hline
  & 54.9 & 3898.3 & 2156.63 & 100\% \\\hline
  & 49.4 & 4236.13 & 2401.5 & 100\% \\\hline
  & 47.9 & 4472.26 & 2475.99 & 100\% \\\hline
\end{tabular}

\vspace{1em}
\underline{NDN file transfer application with AIMD+Hole scheme}\\\\
\begin{tabular}{ | l | l | l | l | l |}
  \hline
  & download time (s) & actual throughput (kbps) & effective
  throughput (kbps) & loss rate \\\hline
  & 42.9 & 5504.04 & 2767.84 & 100\% \\\hline
  & 42.8 & 5779.22 & 2773.88 & 100\% \\\hline
  & 44.6 & 5401.22 & 2659.55 & 100\% \\\hline
  & 46.2 & 5205.68 & 2569.45 & 100\% \\\hline
  & 46.7 & 4937.29 & 2541.65 & 100\% \\\hline
\end{tabular}

\vspace{1em}
\underline{FTP based on TCP/IP}\\\\
\begin{tabular}{ | l | l | l |}
  \hline
  & download time (s) & throughput (kbps) \\\hline
  & 17.58 & 4653.6 \\\hline
  & 17.58 & 4654.4 \\\hline
  & 17.59 & 4656.8 \\\hline
  & 17.58 & 4654.7 \\\hline
  & 17.59 & 4656.4 \\\hline
\end{tabular}

\end{document}
